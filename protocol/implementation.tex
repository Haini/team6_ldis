A naive implementation of the papers approach did not lead to any reproduceable output.
Therefore a list of stepstones that were encountered while working on a solution is
compiled. The most important takeway from the following is that the top and bottom delay
lines need to be symmetric in terms of placement, routing and ports.

The paper authors omit any details on the actual implementation, but as this 
question in the comments of a blog post (most likely from the paper author) shows it was
of interest for them as well \autocite{Mehrdad2010} and that they also struggled with
optimization during synthesis.

The following constraints provide us with a solid ground for further work as placement,
routing and pin assignments are now mostly consistent with every synthesis run. 

\subsection{Unwanted Optimization of Logic Elements}
This was mainly a problem for the \gls{lut} blocks and sometimes a problem for
the d-flipflop. The synthesis tool tries to minimize the use of \gls{lut} blocks as it
rightfully detects the redundancy in the logic. It therefore minimizes the LUT6 (6 inputs)
blocks to combinations of smaller LUT Blocks - controlling the delay is therefore not
possible anymore (on the one hand due to the missing delay of the mutliplexers in the lut
and on the other hand due to the possible asymmetry in the top and bottom delay paths).

\subsubsection{Mitigation}
The Xilinx / Vivado Documentation states on this topic that the DONT_TOUCH attribute is
the best option \autocite{XilinxAttributes}. It should ensure that the signal is kept, in
reality this does not work reliably, or the author didn't understand it fully. 

\begin{lstlisting}[
	style = vhdl,
	caption = {Attribute Declaration in VHDL, for the flipflop},
	label = lst:attrDontTouch,
]
	attribute DONT_TOUCH : string;
	attribute DONT_TOUCH of flipflop: label is "TRUE";
\end{lstlisting}

\subsection{Unwanted Optimization of Logic Element Placement}
The luts would be placed in many different ways for every synthesis run, even if nothing
else changed. Fixing the location of the luts to specific slices solved this problem.
A spreadsheet was created to ease the creation of the assignments.

\subsubsection{Mitigation}
\begin{lstlisting}[
	style = vhdl,
	caption = {Set location of a lut},
	label = lst:setLoc,
]
set_property LOC SLICE_X85Y125 [get_cells gen_bot_coarse_plds[1].i_bot_coarse_pld/g0_b0]
\end{lstlisting}

 
\subsection{Unwanted Optimization of Routing Paths}
As the routing tool is - rightfully - trying to minimize setup and hold times in the
routed design it creates asymmetric paths with delays that can be longer than the
logic delay introduced by the lut blocks. Net Delays of up to 126ns could be observed,
while the logic delay on the same path would only account for 7ns. 
This in itself would not cause a problem if the top and bottom path would show the same
delay. But already a 5\% deviation of top and bottom path would render the delay of
the delay lines useless.

\subsubsection{Mitigation}
Timing analysis and optimization can be disabled for net paths or cells. For the sake of
simplicity all cells in in the delay path very excluded from the timing analysis.
This lead to more direct and more consistent routings. 

\begin{lstlisting}[
	style = vhdl,
	caption = {Disable the timing analysis for luts},
	label = lst:disableTiming,
	]
set_disable_timing [get_cells {gen_top_coarse_plds[*].i_top_coarse_pld/g0_b0}]
set_disable_timing [get_cells {gen_top_fine_plds[*].i_top_fine_pld/g0_b0}]

set_disable_timing [get_cells {gen_bot_coarse_plds[*].i_bot_coarse_pld/g0_b0}]
set_disable_timing [get_cells {gen_bot_fine_plds[*].i_bot_fine_pld/g0_b0}]
\end{lstlisting}

Another option that has to be considered is to route the delay path nets first and only
after that the rest. No specific improvements could be obsereved, alltough other groups
reported some.

\begin{lstlisting}[
	style = vhdl,
	caption = {Route critical nets first},
	label = lst:routeFirst,
	]
set preRoutes [get_nets  {*lut_t_* *lut_b_*}]
route_design -net [get_nets \$preRoutes] -delay
route_design -preserve
\end{lstlisting}

\subsection{Unwanted Optimization of Port Elements}
The synthesis / routing tool can decide how to arrange the port mapping of the luts, e.g.
port A6 of the lut can be mapped to signal I0 while the expected mapping would be A0:I0. 
This again causes asymmetry in the delay paths as certain ports (with A6 being the fastest
and A0 the slowest) of the lut react faster than others (as per the design of the lut).

\subsubsection{Mitigation}
Ports of logic elements can be fixed by the following commands. 

\begin{lstlisting}[
	style = vhdl,
	caption = {Disable the timing analysis for luts},
	label = lst:disableTiming,
	]
set_property LOCK_PINS { I0:A1 I1:A2 I2:A3 I3:A4 I4:A5 I5:A6 } [get_cells *i_top_coarse_pld/g0_b0 ]
set_property LOCK_PINS { I0:A1 I1:A2 I2:A3 I3:A4 I4:A5 I5:A6 } [get_cells *i_top_fine_pld/g0_b0 ]
set_property LOCK_PINS { I0:A1 I1:A2 I2:A3 I3:A4 I4:A5 I5:A6 } [get_cells *i_bot_coarse_pld/g0_b0 ]
set_property LOCK_PINS { I0:A1 I1:A2 I2:A3 I3:A4 I4:A5 I5:A6 } [get_cells *i_bot_fine_pld/g0_b0 ]
\end{lstlisting}
