Some decisions regarding all parts of the implementation were made to provide a
solution that works in simulation at least. These assumptions do not conform to the specification.

\begin{enumerate}
\item Grade of achieveable parallelism is 1
\item Design is not synthesizable and ignores especially the low area criteria
\item Simulation only possible with VHDL-2008 Standard, --std=08 flag mandatory
\end{enumerate}

\subsection{Truncation Function trunc(a)}
The truncation function trunc(a) is implemented as a vhdl function in a package aswell as a
fully fledged component. Our design uses the function, which is located in the
permutate_pkg.vhd, due to its easier usage in sequential code (aka in processes).
\subsection{Permutation Function P}
The permutation function P was also implemented as a vhdl component at first.
Due to the heavy usage of sequential code segments a redesign to a vhdl function was 
done. It resides now in the permutate_pkg.vhd.
The implementation is as suggested by the \autocite{irtf-draft} - but heavily sequential.
One could implement a more elegant solution that allows for parallelization of round
function GB calls that are independet of each other (note that not all calls are
independet, as they operate on the same matrix). 
\subsubsection{Round Function GB}
The round function GB doesn't have its own section but is the foundation of all other
functionalities. 
Yet again it was implemented firstly as a component, only to be translated to a function,
which also resides in the permutate_pkg.vhd. 
It is implemented straight forward as specified.

\subsubsection{Verification}
As the round function represents a vital part of the implementation, the generation of
good test cases was a main goal. To achieve this goal the code of the C reference solution
\autocite{argon2-github}
was analyzed and altered.
The extraction of test vectors at key points of the operation was planned at several parts
of the argon2 implementation, including the following files from the /src directory:

\begin{itemize}
\item ref.c
\item blake2blake2b.c
\item blake2blamka-round-ref.h
\item ..Makefile
\end{itemize}


Extraction of the test vectors from the code was done by simple tagged printfs
of the UINT64 data blocks. The blocks were printed in decimal for simplicity and further
processing (see \Cref{lst:blake2bc}). Removal of the tags was done by hand, a python script (convert_binary.py) then parses the blocks and converts them into their binary representation.

\begin{lstlisting}[
	style = customc,
	caption = {Suboptimal extraction point in blake2b.c},
	label = lst:blake2bc,
]
		...
		...
#define ROUND(r)                       	
    do {                                	
		printf("INPUTSTART\n");			   	
		for (i = 0; i < 16; i++) {		   	
			printf("%" PRIu64 "\n", v[i]); 	
		}								   	
		printf("INPUTEND\n");			   	
		G(v[0], v[4], v[8], v[12]);        
		G(v[2], v[6], v[10], v[14]);    
		...
		...
		printf("OUTPUTSTART\n");			   	
		for (i = 0; i < 16; i++) {		   	
			printf("%" PRIu64 "\n", v[i]); 	
		}								   	
		printf("OUTPUTSTART\n");			   	

\end{lstlisting}

First tests proved to be unsuccessful though, as the implementation of the permutation
function in the C Code differs from the one proposed in the irtf-draft\autocite{irtf-draft}.
To double check on our proposed solution a python script that also implements the
permutation P and round GB function was created.

The python and vhdl implementations did deliver the same output. Analyzing the reference
solution more closely revealed that there are reference, optimized and actually used code
lines that rely heavily on preprocessor statements. 

Analysis of the Makefile showed, that a check for the platform was done (\Cref{lst:makefile}), and if the
platform was supported the optimized version of the files was used.
After modifying the Makefile accordingly (and now actually using the reference solution)
the generated output \textbf{did} match the output of our vhdl solution, see also table
\Cref{tbl:permresults}.

\begin{lstlisting}[
	style = customc, 
	caption = {Parameters for the argon2 execution},
	label = lst:RunArgon,
]
echo -n "password" | ./argon2 somesalt -t 1 -m 16 -p 1 -l 24
\end{lstlisting}

\begin{lstlisting}[
	style = customc,
	caption = {Makefile, deactivate optimizations},
	label = lst:makefile,
]

OPTTEST := $(shell $(CC) -Iinclude -Isrc -march=$(OPTTARGET) src/opt.c -c \
			-o /dev/null 2>/dev/null; echo $$?)
# Detect compatible platform
ifneq ($(OPTTEST), 0)
$(info Building without optimizations)
	SRC += src/ref.c
else
$(info Building with optimizations for $(OPTTARGET))
	CFLAGS += -march=$(OPTTARGET)
	SRC += src/opt.c
endif
\end{lstlisting}

\begin{table}[ht]
	\centering
	\caption{Permutation Function P Outputs}
	\label{tbl:permresults}
	\begin{tabular}{c|cccc}
	\hline
	Input & Python3.5 & VHDL & C Reference & C Actual \\ 
	\hline
	 6A09E667F2BDC948 & 4B86FAA34237F816 & 4B86FAA34237F816 & 4B86FAA34237F816 & 3D9D014CA238A25D \\  
	 510E527FADE682D1 & 826371B4B7CF06DB & 826371B4B7CF06DB & 826371B4B7CF06DB & D9CE83A69663A233 \\
	 6A09E667F3BCC908 & 6915F3A835F68E52 & 6915F3A835F68E52 & 6915F3A835F68E52 & B8023558C91686D7 \\  
	 510E527FADE682E9 & 4A645E346BE317D8 & 4A645E346BE317D8 & 4A645E346BE317D8 &
	 2E207F7532A740EC \\
	 \hline
	\end{tabular}
\end{table}

For one part the upper half of the 128 Byte Input is initialized in an unexpected way that
is not mentioned in the \autocite{irtf-draft}. This seems to be part of the regular way of
implementing blake2b, see the code listing \Cref{lst:blake2bcRound} where v[15] is initialized
by loading a fixed UINT64 and XORing it with an unknown parameter f[1]. 

\begin{lstlisting}[
	style = customc,
	caption = {Differing implementation of Round Function},
	label = lst:blake2bcRound,
]
/*Strange Vector Init*/ v[15] = blake2b_IV[7] ^ S->f[1];
	...
	...
#define G(r, i, a, b, c, d)                                                   
    do {                                                                       
        a = a + b + m[blake2b_sigma[r][2 * i + 0]];                            
        d = rotr64(d ^ a, 32);                                                 
        c = c + d;                                                             
        b = rotr64(b ^ c, 24);                                                 
\end{lstlisting}


When using the correct test vectors from the reference function or our own
generated test vectors from the python script the permutation function shows correct
behaviour.


\subsection{Compression Function G}
\subsection{Argon2, Steps 5-8}


%\begin{lstlisting}[
%	style = vhdl,
%	caption = {},
%	label = lst:,
%]
%\end{lstlisting}
