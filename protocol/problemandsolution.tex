Team 6 was tasked with the following items: 

\begin{enumerate}
\item Implement steps 5-8 of Section 3.2 in the irtf-draft \autocite{irtf-draft}
\item Implement compression function G of Section 3.5 in the irtf-draft \autocite{irtf-draft}
\item Implement permutation P of Section 3.6 in the irtf-draft \autocite{irtf-draft}
\item Implement the trunc(a) function in the irtf-draft \autocite{irtf-draft}
\end{enumerate}

As each sub task depends on its mother task it was decided to proceed in a bottom up
order. 
\\
The truncation function trunc(a) truncates a 64 bit vector to a 32 bit vector and 
is needed to perform multiplications with two 32 bit vectors, which in turn help to
increase the circuit depth of the implementation. This is important when the target
hardware are ASICs or FPGAs.
\\
Permutation function P takes one 128 byte input and applies a round function GB
on it.
\\
Compression function G takes two 1024 byte inputs and applies an XOR, the
permutation function P and another XOR on these inputs.
\\
Steps 5-8 of the Argon2 Operation are related to computing the actual output tag.
They operate on a matrix B[i][j] which is, due to its size, placed
in the DDR2 SDRAM chip. These steps use the compression function G and a variable 
hash function that is provided by Team 5. It also uses the computed indices i and j,
provided by Team 6.
