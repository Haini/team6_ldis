A \gls{trng} based on the paper from \autocite{Majzoobi2011} had to be implemented.

The paper shows an approach that promises a resource efficient and fast solution to the
problem of generating long enough keys for cryptographic algorithms. The reproduction of
the results in the paper is a challenging and important task, as the paper is very
incomplete regarding information about the actual implementation.

The source of randomness is based on the metastability characteristics of a d-flipflop when violating setup or hold times of the flipflop.
To create such violations the novel approach of delay lines is introduced. The data and
clock input ports of the flipflop are both preceded by a series of \glspl{lut} blocks. 
These \glspl{lut} blocks are functionally just used as inverters - but due to the
inner structe of a \glspl{lut} a delay can be introduced by applying different control
signals to the \glspl{lut} ports.
